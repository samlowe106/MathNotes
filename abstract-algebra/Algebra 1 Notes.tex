\documentclass[12pt]{article}
\usepackage{amsmath}
\usepackage{esint} % cyclic integrals
\usepackage{amsfonts} % \mathbb
\usepackage{mathtools}
\usepackage{latexsym}
\usepackage{amssymb} % therefore
\usepackage{hyperref} % allows hyperlinks, including in the table of contents
\usepackage{amsthm}
%\usepackage[document]{ragged2e} % left justifies text

\setlength{\parindent}{0em}
\setlength{\parskip}{1em}

\DeclareMathOperator{\Div}{Div}
\DeclareMathOperator{\Curl}{Curl}
\DeclareMathOperator{\dom}{dom}
\DeclareMathOperator{\E}{E}
\DeclareMathOperator{\Var}{Var}
\DeclareMathOperator{\Cov}{Cov}
\DeclareMathOperator{\Corr}{Corr}

\newcommand{\N}{\ensuremath{\mathbb{N}}}
\newcommand{\Z}{\ensuremath{\mathbb{Z}}}
\newcommand{\Q}{\ensuremath{\mathbb{Q}}}
\newcommand{\R}{\ensuremath{\mathbb{R}}}
\newcommand{\C}{\ensuremath{\mathbb{C}}}

\newcommand{\Laplace}[1]{\ensuremath{\mathcal{L}\left( #1 \right)}}

\newcommand{\Lemma}{\textbf{Lemma}}

\newcommand{\Thrm}[1]{\textbf{Theorem (#1)}}

\newcommand{\Def}[1]{\textbf{Definition (#1)}}

\newcommand{\Example}{\textbf{Example}}

%%%%%%%%%%%%%%%%%%%%%%%%%%%%%%%%%%%%%%%%%%%%%%%%%%%%%%%%%%%%%%%%%%%%%%%%%%%%%%%%
% Felipe Portales' HomeWorks
% Maintainer: Felipe Portales-Oliva (f.portales.oliva@gmail.com)
%
% Document class to produce simple yet stylish homework submissions
%
% This package is public domain, according to :
%    The Unlicense <https://unlicense.org>


% CERTAIN CHANGES HAVE BEEN MADE TO SPECIALIZE THIS DOCUMENT CLASS TO
% A COMPUTER SCIENCE CONTEXT (HENRY BURTIS, MAY 2021)

\newcommand{\set}[1]{\ensuremath{\left\lbrace  #1 \right\rbrace}}
\newcommand{\setbuild}[2]{\ensuremath{\left\lbrace  #1 \;\middle|\; #2 \right\rbrace}}
\newcommand{\powset}[1]{\ensuremath{\mathcal{P}}(#1)}
\newcommand{\func}[3]{\ensuremath{#1 \;:\; #2 \rightarrow #3}}


\title{Template}
\author{Sam Lowe}

\begin{document}

\maketitle

\tableofcontents

\pagebreak

\section{Section 1}

\underline{Lecture 1}

Motivation: how do we solve polynomials? In solving the polynomial $9x^2 - 4 = 0$, we can add, subtract, multiply, divide, and take roots.

Field extension

The solution to our polynomial came in pairs; $\pm 2/3$. Notice the symmetry that these roots come in pairs.

Q sits inside R which sits inside C. However, "between" Q and R, we have $\Q(\sqrt{2})$. We can also thing of the field $\Q(i)$ as sitting between Q and C.

In general, we can solve quadratics with the quadratic formula $$x = \frac{-b \pm \sqrt{\Delta}}{2a}$$ where $\Delta = b^2 - 4ac$. Notice that again we have symmetry that interchanges these two square roots.

Degree 2 polynomials are solvable in radicals; all we have to do in addition to solving degree 2 polynomial equations over a field, we only have to include additional numbers - roots - of the form $\sqrt{b^2 - 4ac}$. Here, we're including \textit{roots of numbers that we already have}.

Cardano: $x^3 + px + q = 0$ $$\sqrt[3]{\alpha + \sqrt{\beta}} + \sqrt[3]{\alpha - \sqrt{\beta}}$$ where $\alpha = \frac{-q}{2}$ and $\beta = \frac{q^2}{4} + \frac{p^3}{27}$.

Theorem (Abel-Ruffini) For any $n \geq 5$ there is a polynomial of degree $n$ that is not solvable in radicals.

One example is $x^5 - x - 1 = 0$.

This theorem does \textbf{not} say there are no solutions, or that no polynomials of degree 5 or greater cannot be solved by radicals. It only says that those roots do not conform to a certain form.

The idea behind this proof is: \begin{enumerate}
    \item Solvability in radicals means that we give ourselves (``pass to a field extension'') a finite sequence of $n$th roots. For example, in the quadratic example, we went from $\Q \rightarrow \Q(\sqrt{\Delta})$. In the cubic example, we went from $\Q \rightarrow \Q(\sqrt{\beta}) \rightarrow \Q(\sqrt{\beta}, \sqrt{\alpha})$
    \item These roots have symmetries: $\sqrt{\Delta} \rightarrow -\sqrt{\Delta}$, or $\sqrt[3]{\alpha + \sqrt{\beta}} \rightarrow \sqrt[3]{\alpha + \sqrt{\beta}}e^{2\pi i/3}$ (note the third root of unity!)
    \item So we can show that a polynomial is not solvable in radicals by showing that the symmetries of its roots are not of this iterated, cyclic nature.
\end{enumerate}


A \textbf{ring} is a set $R$ equipped with two binary operations, denoted $(a, b) \rightarrow a + b$ (addition) and $(a, b) \rightarrow a \times b = a \cdot b = ab$ (multiplication).

Both of these operations are associative $a(bc) = (ab)c$ and unital ($a + 0 = a$, $1 \times a = a \times 1 = a$).

Addition is commutative and has inverses; $a + b = b + a$ and $a + (-a) = 0$.

Multiplication distributes over addition on both sides $c(a + b) = (a + b)c = ac + bc$.

A ring $R$ is called commutative if its multiplication operation is commutative.

Examples:

$\Z$ is a ring with the usual addition and multiplication.

$\Z/n\Z$ is a ring with addition mod $n$ and multiplication mod $n$. ($a \equiv b mod n$ if $n \mid b - a$)

Exercise: these are well-defined and make $\Z/n\Z$ a ring.

Example: $\Z/1\Z$ is the singleton ring.

Exercise: $0r = 0$ for all $r \in R$, $R$ a ring, implies that $0 = 1$ iff $R = \{ 0 \}$.

Example: $\Z/2\Z = \{ 0, 1 \}$

Example: $\Z/3\Z = \{ 0, 1, 2 \}$

Example: $\Z/4\Z = \{ 0, 1, 2, 4 \}$

Interestingly, in $\Z/4\Z$, $2 \times 2 = 0$, so it fails to have multiplicative inverses.

A commutative ring with $0 \neq 1$ is a field if it has multiplicative inverses for nonzero elements. $\Z, \{ 0 \}$, and $\Z/4\Z$ fail to be fields.

$\Q$, $\R$, and $\C$ are fields.

The set $M_n(\R)$ of $n \times n$ matrices a non-commutative ring for $n \geq 2$ under entrywise addition and standard matrix multiplication. (The entries can come from any set, not just from $\R$.)

Proposition $\Z/n\Z$ is a field iff $n$ is prime.

Lemma Given $m \in \Z$, $\overline{m}$ has a multiplicative inverse in $\Z/n\Z$ if and only if $m$ and $n$ are relatively prime.

In a field, $ab = 0$ if and only if $a$ or $b$ is 0.

Conversely, if $m$ and $n$ are relatively prime, then $rm + sn = 1$ for some integers $r$ and $s$ (Bézout's identity), so $rm \equiv 1 mod n$, i.e. $\overline{r} \cdot \overline{m} = 1$.

Example $F_4 = \{ 0, 1, \omega, \overline{\omega} \}$

$1, \omega, \overline{\omega}$, are roots of $x^3 - 1$.






\end{document}
