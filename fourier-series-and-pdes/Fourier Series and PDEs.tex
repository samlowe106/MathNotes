\documentclass[12pt]{article}
\usepackage{amsmath}
\usepackage{esint} % cyclic integrals
\usepackage{amsfonts} % \mathbb
\usepackage{mathtools}
\usepackage{latexsym}
\usepackage{amssymb} % therefore
\usepackage{hyperref} % allows hyperlinks, including in the table of contents
\usepackage{amsthm}
%\usepackage[document]{ragged2e} % left justifies text

\DeclareMathOperator{\Div}{Div}
\DeclareMathOperator{\Curl}{Curl}

\DeclareMathOperator{\dom}{dom}

\newcommand{\N}{\mathbb{N}}
\newcommand{\Z}{\mathbb{Z}}
\newcommand{\Q}{\mathbb{Q}}
\newcommand{\R}{\mathbb{R}}
\newcommand{\C}{\mathbb{C}}

\title{Fourier Series and PDEs Notes}
\author{Sam Lowe}

\begin{document}

\maketitle

\tableofcontents

\pagebreak

% Office Lake 538, W 12-1pm in person, Tues 9-11am on Zoom


\section{Review of 2341 and Preliminaries}

\underline{Lecture 1}

We can write an ordinary differential equation to find the temperature at any point on a rod of length $L$; $T(x)$ could represent the temperature, where $x$ is length. However, we know that the temperature at each point also depends on time, so we have to expand our model to cover two variables.

We need partial derivatives. For our temperature function $T = f(x, t)$, we'll be concerned with $\frac{\partial T}{\partial x}$ and $\frac{\partial T}{\partial t}$.

We have \textbf{heat equation in one dimension}, $$\frac{\partial T}{\partial t} = \frac{\partial^2 T}{\partial x^2}$$ which can also be written as $T_t = T_{xx}$. This equation has \textit{no general solution.} However, we can solve it using Fourier series.

\textbf{Wave equation in one dimension} measures displacement $u(x, t)$

$$\frac{\partial^2 u}{\partial t^2} = c^2\frac{\partial^2 u}{\partial x^2}$$

\textbf{Maxwell's equation}, which is used in communication.

\textbf{Navier-Stokes equation} which is used in weather forecasting.

In ordinary differential equations, a \textit{general solution always exists}. In PDEs, there are \textit{no general solutions}.

\subsection{Review of MATH 2341}

\textbf{First order separable differential equations} take the form $$\frac{dy}{dx} = f(x)g(y)$$ with solution $$\int \frac{1}{g(y)}dy = \int f(x)dx$$ yielding $G(y) = F(x) + C$ which is the implicit general solution, and can typically be rearranged to isolate $y$ and yield an explicit general solution.

Example: $y' = xy$ which yields the explicit solution $|y| = e^{\frac{x^2}{2} + C}$, typically written as $Ce^{\frac{x^2}{2}}$, so $y$ equals either $Ce^{\frac{x^2}{2}}$ or $-Ce^{\frac{x^2}{2}}$.

We also have \textbf{first-order linear differential equations} (some of which are separable!). $y' + p(x)y = q(x)$. We find the integrating factor $I(x) = e^{\int p(x)dx}$. Multiplying both sides by $I(x)$ yields $I(x)y + p(x)I(x)y = I(x)q(x)$; the left side resembles the product rule of $\left(I(x)y\right)'$. So we integrate $I(x)y = \int I(x)q(x)dx + C$.

Example: $y' - \frac{1}{x}y = 2x + 1$ for $x > 0$. We set $p(x) = -\frac{1}{x}$ and get $I(x) = e^{-\ln(x)} = \frac{1}{x}$. We then have $\frac{1}{x}y' - \frac{1}{x^2}y = \left( \frac{1}{x}y \right)' = 2 + \frac{1}{x}$. Integrating yields $\frac{1}{x}y = 2x + \ln(x) + C$ so $y = 2x^2 + x\ln(x) + Cx$.

\textbf{Second order constant-coefficient differential equations} take the form $ay'' + by' + cy = 0$ for $a \neq 0$. The solution is $y = e^{rx} \to y' = re^{rx} \to \to y'' = r^2e^{rx}$ where each $r$ is the roots of the characteristic equation (polynomial) $at^2 + bt + c = 0$. \begin{itemize}
    \item If $b^2 - 4ac > 0$, $r_1 \neq r_2$ and both are real. Our solution is $y = C_1e^{r_1x} + C_2e^{r_2x}$, which we call a \textbf{superposition}
    \item If $b^2 = 4ac$ then $r_1 = r_2$ and our solution is $y = C_1e^{r_1x} + xe^{r_1x}$
    \item Finally, if $b^2 - 4ac < 0$, both $r_1$ and $r_2$ are complex. Setting $q = 4ac - b^2$, our solution take the form $$y(x) = e^{\beta x} \left[ c_1 \cos(qx) + c_2 \sin(qx) \right]$$ If $p > 0$, the sine wave's amplitude increases; if $p < 0$, it decreases.
\end{itemize}


We also should summarize some basic facts about the trigonometric functions: \begin{itemize}
    \item $\sin(x)$ has period $2\pi$ radians with zeroes located at $n\pi$ for $n \in \Z$.
    \item In general, $\sin(\beta x)$ and $\cos(\beta x)$ have period $\frac{2\pi}{\beta}$.
    \item For $L > 0$ and real, if we have $\sin(\frac{n\pi x}{L})$ and $\cos(\frac{n\pi x}{L})$, the period is $\frac{2\pi}{\frac{n\pi}{L}} = \frac{2L}{n}$ so $2L$ is the common period for any function of this form.
\end{itemize}

If we want to compute $$I = \int_{-L}^L \cos \left(\frac{n\pi x}{L}\right) \cos \left( \frac{m\pi x}{L} \right) dx$$ integrating by parts\footnote{$\int u dv = uv - \int v du$} with $$\begin{aligned}
    u  =& \cos \left( \frac{n\pi x}{L} \right)\\
    dv =& \cos \left( \frac{m\pi x}{L} \right)\\
    du =& -\frac{n \pi}{L}\sin \left( \frac{n\pi x}{L} \right)\\
    v  =& \int \cos \left( \frac{m \pi x}{L} \right) dx = \frac{L}{m \pi} \sin \left( \frac{m \pi x}{L} \right)
\end{aligned}$$

$$\begin{aligned}
    I&=uv \bigg\rvert_{-L}^L -\frac{L}{m\pi}\int_{-L}^L \sin \left(\frac{m\pi x}{L} \right) \frac{-n\pi}{L} \sin \left( \frac{m\pi x}{L} \right) dx\\
    &= \frac{n}{m} \int_{-L}^L \sin \left( \frac{n\pi x}{L} \right) \sin \left( \frac{m\pi x}{L} \right) dx = \frac{n}{m} \int_{-L}^L u dv\\
    &= \frac{n^2}{m^2} \int_{-L}^L \cos \left( \frac{n\pi x}{L} \right) \cos \left( \frac{m\pi x}{L} \right) = \frac{n^2}{m^2}I
\end{aligned}$$ So $I = \frac{n^2}{m^2} I$ implying $(m^2 - n^2)I = 0$. Either $m = n$ or $I = 0$, so $$\int_{-L}^L \cos\left( \frac{n\pi x}{L} \right) \cos\left( \frac{m\pi x}{L} \right) = 0$$ for $n = 1, 2, 3, \ldots$, $m = 1, 2, 3, \ldots$, and $m \neq n$

\section{Fourier Series}

$$f(x) \sim \frac{a_0}{2} \sum_{n=1}^{\infty} $$

\section{Sturm-Liouville Problems}

\subsection{Four Standard Cases}

\begin{enumerate}
    \item
    \item
    \item
    \item
\end{enumerate}

\section{The 1-D Homogeneous Heat Equation}

$$u_{t} = ku_{xx}$$

\subsection{Derivation}

\subsection{Solution}

\section{n-Dimensional Heat Equation}

$$u_t = k(\Delta u)$$ where $\Delta$ is the Laplacian operator given by $$(\Delta v)(x_1, ..., x_n) = \sum_{i = 1}^n v_{x_ix_i}$$

\section{The Wave Equation}

\subsection{1-Dimensional Case}

$$u_{tt} = c^2u_{xx} + q(x,t)$$

\subsubsection{Derivation}

\subsubsection{Homogeneous Solution}

When $q(x,t)$ is identically zero, we have the homogeneous equation $$u_{tt} = c^2u_{xx}$$

\subsubsection{Non-homogeneous Solution}

When $q(x,t)$ is not identically zero, the solution to
$$u_{tt} = c^2u_{xx} + q(x,t)$$ is

\subsection{On a Circular Membrane}

\subsubsection{Solution}

\subsection{n-Dimensional Case}

$$u_{tt} = $$

\subsubsection{Solution}


\section{The Laplace Equation}

\subsection{2-Dimensional Case}

$$u_{xx} + u_{yy} = 0$$

\subsubsection{Derivation}

\subsubsection{Homogeneous Solution}

\subsubsection{Non-homogeneous Solution}

\subsection{n-Dimensional Case}

\subsubsection{Solution}

\section{Other Equations}

\subsection{Brownian Motion}

$$u_t(x, t) = au_{xx}(x,t) - bu_x(x, t)$$

\subsection{Diffusion-Convection Problems}

$$u_t(x, t) = ku_{xx}(x,t) - au_x(x, t)  bu(x,t)$$

\subsection{Black-Scholes Equation}

$$V_t(S, t) + \frac{1}{2}\sigma^2S^2V_{SS}(S,t) + rSV_S(S,t) - rV(S,t) = 0$$

\subsection{Schrödinger Equation}

Evolution of a Quantum State
$$ih\psi_t(x,t) = -\frac{\hbar^2}{2m} $$

\subsection{Klein-Gordon Equation}

Motion of a Quantum Scalar Field

$$\psi_{tt}(x,y,z,t) = c^2\Delta \psi(x,y,z,t)-\frac{m^2c^4}{\hbar^2}\psi(x,y,z,t) $$

With $i^2 = -1$ and $c$ being the speed of light. In one dimension,

$$u_{tt}(x,t) = c^2u_{xx}(x,t) - au(x,t)$$

for constant $a > 0$.

\subsection{Telegraph Equation}

$$au_{tt}(x,t) + bu_t(x,t) + cu(x,t) = u_{xx}(x,t) $$

\subsection{Dissipative Waves}

$$ u_{tt}(x,t) + au_t(x,t) + bu(x,t) = c^2u_{xx}(x,t) - du_x(x,t) $$

$a, b, d \geq 0$ not all 0 and $c > 0 $

\subsection{Transverse Vibrations of a Rod}

$$ u_{tt}(x,t) + c^2u_{xxxx}(x,t) = 0$$

\subsection{Helmholtz Equation}

$$ \Delta u(x,y,z) + k^2u(x,y,z) = 0 $$

$$ \Delta u(x,y,z) - k^2u(x,y,z) = 0 $$

\subsection{Steady-state Connective Heat}

$$ \Delta u(x,y) - au_x(x,y) - bu_y(x,y) + cu(x,y) = 0 $$

\subsection{Plane Problems in Continuum Mechanics}

$$ \Delta\Delta u(x,y) = u_{xxxx}(x,y) + 2u_{xxyy}(x,y) + u_{yyyy}(x,y) = 0$$

\subsection{Euler-Tricomi Equation}

Plane Transonic Flow

$$ u_{xx}(x,t) = xu_{xx}(x,t) $$

where $u(x,t)$ is a function of speed.

\subsection{Fisher Equation}

Advance of advantageous genes in a population

$$ u_t(x,t) = Du_{xx}(x,t) + ru(x,t)(1-u(x,t)) $$

\subsection{Boussinesq Equation}

Fluid dynamics

$$ \eta_{tt}(x, t) - gh\eta_{xx}(x,t) - gh\left( \frac{2}{3h}\eta^2(x,t) + \frac{h^2}{3}\eta_{xx}(x,t) \right)_{xx} = 0  $$

\subsection{Navier-Stokes Fluid Flow Equations}

The Navier-Stokes fluid-flow equations are the system of partial differential equations given by $$\begin{aligned}
    \nabla \cdot u &= 0\\
    \rho \left(\frac{\partial u}{\partial t} + (u \cdot \nabla)u \right) &= -\nabla p + u\nabla^2 u + \rho F
\end{aligned}$$ which model every known fluid. The first equation in the system represents the conservation of mass, and the second represents the conservation of momentum.


\textbf{Millenium Problem (Navier-Stokes Existence and Smootheness)} Prove or give a counterexample to the following statement: in three space dimensions and time, given an initial velocity field, there exists a vector velocity and a scalar pressure field, which are both smooth and globally defined, that solve the Navier-Stokes equations.

% The Clay Mathematics Institute offers a US\$1,000,000 prize  Before the Clay Mathematics Institute will consider a proposed solution, all three of the following conditions must be satisfied: \begin{enumerate}
%    \item the proposed solution must be published in a Qualifying Outlet,
%    \item at least two years must have passed since publication, and
%    \item the proposed solution must have received general acceptance in the global mathematics community
%\end{enumerate}



\section{The Laplace Transform}

The Laplace transform can be used to ``suppress'' the time variable $t > 0$ in a PDE; for one-dimensional equations, this yields an ordinary differential equation in $U(x,s)$ where $s$ is a constant.

\section{The Fourier Transform}

The Fourier transform can be used to ``suppress'' spatial variables $-\infty < x, y, z, \dots < \infty$ much the same way that the Laplace transform ``suppresses'' the time variable.

\section{The Method of Green's Functions}

\end{document}