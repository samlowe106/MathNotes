\documentclass[12pt]{article}
\usepackage{amsmath}
\usepackage{amsfonts} % \mathbb
\usepackage{mathtools}
\usepackage{latexsym}
\usepackage{amssymb} % therefore
\usepackage{hyperref} % allows hyperlinks, including in the table of contents

\title{Multivariable and Vector Calculus Notes}
\author{Sam Lowe}

\begin{document}

\maketitle

\tableofcontents

\pagebreak


A \textbf{Finite State Machine} or \textbf{Finite Automaton} is a 5-tuple $(Q, \Sigma, \delta, q_0, F)$ where \begin{itemize}
    \item $Q$ is a finite set called the \textbf{states},
    \item $\Sigma$ is a finite set called the \textbf{alphabet},
    \item $\delta : Q \times \Sigma \to Q$ is the \textbf{transition function},
    \item $q_0 \in Q$ is the \textbf{start state}, and
    \item $F \subseteq Q$ is the set of \textbf{accept states} (also known as \textbf{final states})
\end{itemize}

If $A$ is the set of all strings that $M$ accepts, we say that $A$ is the \textbf{language of machine} $M$ and denote it $L(M)=A$. We say that $M$ \textbf{accepts} $A$ or $M$ \textbf{recognizes} $A$ (we prefer the latter term to avoid confusion when referring to strings and languages). Machines may accept many strings, but recognize only one language.

A language is called a \textbf{regular language} if some finite automaton recognizes it.

Let $A$ and $B$ be languages. We define the regular operations union, concatenation, and star as follows: \begin{itemize}
    \item \textbf{Union}: $A \cup B = \{ x \, | \, x \in A \text{ or } x \in B \}$
    \item \textbf{Concatenation}: $A \circ B = \{ xy \, | \, x \in A \text{ and } x \in B \}$
    \item \textbf{Star}: $A^* = \{ x_1x_2\ldots x_k \, | \, k \geq 0 \text{ and each } x_k \in A \}$ 
\end{itemize}



\end{document}